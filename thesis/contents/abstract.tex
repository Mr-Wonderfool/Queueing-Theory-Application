\begin{abstract}
本文旨在探究校门开放方式对学生通行效率的影响。
通过排队论的方法,建立了三种不同复杂度的模型来分析和比较两种开门模
式:默认开门和默认关门。首先,我们简化了模型,假设非高峰期单校门情
形下的随机服务时间为固定数值,构建了模型1。接着,模型2引入了负指数
分布来更细致地描述服务时间。最后,模型3综合考虑了高峰期和非高峰期,
多校门情形以及顾客流在不同队列中的转移概率,是最符合实际的模型。

为了构建排队论基本模型,对服务规则与顾客源建立随机性模型。
在服务规则分析中,我们考虑了刷卡时间和经过时间的波动性,并采用负
指数分布来描述校门的服务过程。顾客源分析则区分了学生和其他人员,分
别建立了适用于高峰期和非高峰期的泊松流模型。此外,还考虑了步行与单
车两种不同的出行方式,并对刷卡成功率进行了建模。

理论分析部分,我们使用了排队论的基本公式和概率论,得到了描述系统运
行特征的差分方程,然后求得了稳态解,并通过状态转移图的方式进行可视化。
对于不同的模型,分别通过Pollaczek-Khintchine公式
和Little法则,计算了平均队长和平均等待时间,并将数值模拟时
用到的参数代入公式。定性的来说,各模型中均可以表示默认关门的通行效率
要低于默认开门的模式。定量的来说,对于单通道模型下的默认开门的平均队长
与默认关门的平均队长的比值,确定服务型为0.52,随机服务型为0.42.

数值模拟部分,我们采用了元胞自动机模型和蒙特卡洛法,对排队-通行过程进行了精细模拟
,并进行了统计分析以验证理论解的正确性。元胞自动机分别模拟单通道与多通道
的情况,在多通道时,设置转移概率,有效模拟真实过程,并进行可视化展示。
统计结果时,使用10000个时间步长下的平均队列作为评价指标,
模拟100次,对结果进行统计分析,最后在$95\%$的置信度下,
给出两者平均队长之比的置信区间为[0.4516,0.4526],与理论结果的定性结果一致,
在具体的定量分析上有有一定差距。
\par 考虑实际情况,在单通道随机模型的基础上建立多通道随机模型,重要的特征是
行人可以以一定的概率换道,从而更快的通过校门。在高人流量和低人流量的情况下分别对
单通道和多通道模型进行数值模拟,得到平均队长。模拟结果显示,多通道可以显著减少
平均队长,同时在多通道、成功率较高的情况下,两种开门方式下的通行效率相差不大。
考虑到校园管理对于安全性的需求,结合模拟结果,给出不同场景下的开门方式建议。
\par 最后,在模型建立完成的基础上对模型进行评价与展望。
本文讨论了模型的优缺点,并提出了改进方向,包括实地调研获取参数、选择更
符合实际的模型以及将模型推广到更多场景。

\keywords{排队论\quad 通行方式 \quad 元胞自动机 \quad 泊松过程}
\end{abstract}