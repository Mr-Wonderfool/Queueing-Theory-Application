\section{问题分析}
\subsection{建模基本思路}
从理论的角度,通过排队论的方法对问题进行建模。需要考虑到服务对象,服务规则来构建排队模型,使用
常见的系统平均队长$ L_s$,平均等待时间$W_s$,服务强度$\rho$等概念来对比两种开门模式的效率。
另一方面,从模拟的角度,基于理论分析的顾客、服务模型,建立元胞自动机模拟
排队过程,计算最终结果,与理论计算值进行对比。
\par 由于两个开门模式的差别在于服务模式的不同,而对于顾客源,两者是一致的。
因此,可以将顾客源进行适当的简化,而突出服务模式的对排队系统的影响。
\par 另一方面,研究由浅入深,建立三个模型,模型的复杂度逐步增加,效果更加接近实际:
\begin{description}
    \item[模型1] 在时间上,只考虑非高峰期,在通道上只考虑一个校门的情形,在服务系统方面将随机服务时间用
    固定数值(均值)代替,从而简化模型。
    \item[模型2] 在时间上,只考虑非高峰期,在通道上只考虑一个校门的情形,在服务系统方面,随机服务时间用
    负指数分布来描述,从而细化对服务系统的刻画。
    \item[模型3] 在时间上,既考虑非高峰期,又考虑高峰期,在通道上考虑多个校门的情形,并且
    顾客流可以在不同队列中按照一定概率转移,在服务系统方面,随机服务时间用
    负指数分布来描述,得到最终的效率对比结果。
\end{description}
以下对这些问题一一建模。

\subsection{服务规则分析}
	在两种方案中,无论默认状态是开门还是关门,顾客的服务时间都是相对固定的。

    对于默认关门的情况,顺利刷卡的服务时间为刷卡时间$t_{g}$,开门时间$t_{o}$,经过时间$t_{p}$
    与关门时间$t_{c}$组成。其中开门、关门时间即为定值,而且是相等的,
    相加作为固定服务时间$2t_{c}$。而刷卡时间、经过时间有一定的波动,
    对于不同年龄、不同身份、不同交通工具的人有较大的差异。
    例如,骑行单车入校的人由于要控制单车,取出与放回校园卡
    的时间比步行要长;年龄较大的人通过时间比年轻人要长;通过时看手机
    的人,通过时间比正常通行的人要长。将这两个服务时间之和即为随机服务时间$t_{r}$
    
    但总的来说,校门的随机服务时间有一定的共性。首先,服务时间必然是正数;其次大多数人的
    服务时间都较为稳定,服务时间极长的人很少。为了描述这种变化,使用常见的负指数分布
    来描述校门的服务过程。

    最后将固定服务时间与随机服务时间相加即为总服务时间$t_{all}=2t_{c}+t_{r}$.

    而对于刷卡失败的情况,在非高峰期时,队列稀疏的情况下,对正常队列的影响不大,
    但是在高峰期时,由于进出压力大,人们在刷卡的流程中形成惯性,而刷卡失败会
    导致队列结果的破坏与前后人流的拥堵,从而大大降低通过效率,
    甚至导致整条队列的长期停滞。对刷卡失败的情况,需要细致的建模。

    对于默认开门的情况,顺利刷卡的服务时间为刷卡时间与经过时间,也可以视为定值。
    但是对于刷卡失败的人员,需要在保安的协助下进行登记之后才能入内,而下一个
    人的经过时间需要加上开门时间。

\subsection{顾客源分析}
在我们的问题中,服务对象为进出校的学生以及其他人员。在这些服务对象中,又分为步行
与单车两重出行方式。在分布时间上,又分为高峰期与非高峰期。总的来说需要对不同时期、不同方式、
不同身份的服务对象分别进行建模,以下是具体分析过程

\subsubsection{学生}
	学生群体是所有服务对象中规模最大,成分最复杂的情况,也是开门方式能够影响的主要人员,需
    要进行详细的建模。

	从时间分布的角度来说,由于学生群体在高峰期与非高峰期上具有明显的差异。在非高峰期时,学生
    进出校门的时间随机性强,且较为稀疏。进一步分析,在不相互重叠的时间段内顾客到达的数量
    是相互独立的;且在任意时间段内,到达的概率与时间无关,因为在非高峰期的出行没有显著规律;
    最后,由于非高峰期人群的通行具有稀疏性,所以在短时间内有多人进入的概率极小。
    
    综合以上原因,顾客源适合用标准泊松流来建模。而且由于几乎不存在过于拥挤的
    可能,因此也不考虑容量限制。即$M/G/1/\infty/\infty/FCFS$模型。

    在高峰期,人员流动迅速,排队长度较长,但是高峰期的流通人数是由限制的,即顾客源的容
    量是有限的,将此时顾客源的容量记为$C_{source}$。而且在校门口排队的空间容量是有限的,
    在高峰期容易被填满,将此时的总容量记为$C_{queue}$。但总体来说,顾
    客到来的间隔时间还是可以服从一个均值较小的泊松分布,这个均值比非高峰期要小得多。综合
    起来,高峰期的学生进校排队模型可以简化为$M/G/3/C_{queue}/C_{source}/FCFS$
    此外,由于步行与单车在排队是占据的空间有多不同,所以最后的空间限制并非直接作用于人数,
    而是需要通过作用于空间,间接控制人数。

    从刷卡成功率的角度分析,学生群体的刷卡成功率很高。

\subsubsection{其他}
    对于其他人群,在进校时间上没有明显的高峰期与非高峰期之分,所以用统一的泊松过程模拟即可。
    而且由于其他人群的进出校需求与非高峰期的学生大致相当,可以直接与学生统一。
    对于交通方式来说,主要是步行,而骑单车的人数较少,故简化为全步行。
    在刷卡成功率方面,这个群体的成功率要比学生群体更低。如果与学生顾客源
    统一,则需要按照双方人数加权平均,得到综合的刷卡成功率。