\section{模型假设}
\begin{enumerate}
    \setlength{\leftskip}{0pt}
    \item \textbf{服务对象由行人和自行车组成,自行车在排队时会下车推行}。
    \\ 根据实际生活经验,排队人群主要由行人和自行车组成,同时自行车多会下车推行,从而在校门前
    行人和自行车的速度可以认为一致。同时由于自行车和行人占用的空间大小不同,在问题求解过程中将两者
    分开建模。
    \item \textbf{行人在排队时如果发现前方队伍发生阻塞,有一定概率换道,这个概率对所有的行人相同,但是自行车
    不会换道}。
    \\ 行人为了更快通过校门,在发现前方队伍发生阻塞时往往会进行换道。在满足换道条件时(建模过程中
    将会定量说明)会有一定的概率换道。但是自行车由于所占空间较大,往往不会换道,所以这里为了简化模型,
    假设自行车不会换道。
    \item \textbf{刷卡失败后,默认该对象离开队伍,重新排队后刷卡进校}。
    \\ 在行人刷身份证进校时,由于不知道在哪里刷卡,往往会花费较长的刷卡时间,失败的概率也较高。观察到
    这样的行人一般会在安保人员的引导下先离开队伍,等到队伍有空位时再次刷身份证进校。所以如果刷卡失败,假设
    行人会先离开队伍,重新排队后再次刷卡入校。
\end{enumerate}