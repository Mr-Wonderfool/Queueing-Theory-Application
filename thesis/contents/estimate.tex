\section{模型评价}
\subsection{模型优缺点}
\subsubsection{模型优点}
\begin{itemize}
    \item 使用排队论中的经典理论,先得到理论结果,对两种模式做出定性与定量的比较。
    \item 使用元胞自动机模型对排队-通行的过程进行精细模拟,
    考虑不同时间段(非高峰期与高峰期),不同形式(单通道与多通道)。
    \item 使用蒙特卡洛法进行模拟,对结果进行统计分析,从而验证理论解的正确性。
    \item 对排队过程进行可视化分析,使得排队过程得以展现,从而直观地观察到不同开门模式的差异性。
\end{itemize}
\subsubsection{模型缺点}
\begin{itemize}
    \item 使用的是排队论中的基础模型,虽然经分析可知模型可以基本符合
    改场景的情况,但过于理想化的模型在定量描述方面有一定缺陷。
    \item 在统计学意义上,样本数量不够大,如果增加模拟次数可以使结果更加准确。
    \item 理论解过于理想化,求解的结果单一,容易与实验结果发生偏差,对于这种
    随机服务系统,应该按照其随机性的规律,确定结果范围,防止出现过大偏差。
\end{itemize}
\subsection{改进方向}
\begin{itemize}
    \item 实地调研,统计得到对应参数的参考值,获取各项参数的统计规律。
    \item 深入研究排队论,选择更为符合实际统计规律的概率模型,提高定量分析的准确度。
    \item 将模型推广到更多使用场景,如地铁站出入口,对于不同场景建立恰当的
    排队论模型与元胞自动机模型,进行理论分析与数值模拟。
\end{itemize}