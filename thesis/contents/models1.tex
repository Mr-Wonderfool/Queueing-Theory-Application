\section{模型建立与求解}
\subsection{基本理论}
\subsubsection{服务系统的规则设定}
    首先从理论上,基于排队论的基本公式与概率论相关知识,在理论上对两种开门模型
    进行理论计算,得到平均队长与平均等待时间作为评价通行效率的指标。

    从顾客源的角度来说,顾客源的输入为泊松过程,泊松过程的概率表达式如下

    \begin{equation}
        P_n(t)=\frac{e^{-\lambda t}(\lambda t)^n}{n!},t>0,n \in \mathbb{N}
    \end{equation}
    $\lambda$表示单位时间内平均到达的顾客数,
    对于非高峰期与高峰期的顾客源,分别使用不同的$\lambda_l,\lambda_h$来描述。

    按照泊松过程的概率公式,随机生成顾客队列,
    并让队列以$v_f=1 m/s$的速度前进。

    对于到达校门的人,开始进行服务。服务时间由固定服务时间与随机服务时间相加得到,
    即$t_{all}=2t_{c}+t_{r}$。$t_r$服从负指数分布,概率密度与分布函数如下:
    \begin{equation}
        \begin{aligned}
            f(t) &=\mu e^{-\mu t},  \\
            F(t) &=1-e^{-\mu t},t>0
        \end{aligned}
    \end{equation}

    求的随机服务时间$t_r$的数学期望$E(t_r)=\frac{1}{\mu}$即为平均随机服务时间,
    因此$\mu$的意义就是单位时间内的平均服务人数,也就是校门的平均通过速率。
    在加上固定服务时间$2t_c$,得到总服务时间的数学期望$E(t_{all})=2t_c+\frac{1}{\mu}$,
    方差$ Var(t_{all})=Var(t_r)=\frac{1}{\mu^2}$。
    对于精度要求不高的模型,也可以使用一个常数(平均服务时间)来代替,
    即数学期望$E(t_{all})=2t_c+\frac{1}{\mu}$,
    方差$ Var(t_{all})=0$。

    有了顾客源与服务时间的随机分布模型,可以服务定义强度
    $\rho=\lambda E(t_{all})=2t_c \lambda+\frac{\lambda}{\mu}$

    在这里假设服务强度$\rho<1$,而这个假设也是合理的,因为否则会导致队伍长度发散。
    使用排队论中的经典公式Pollaczek-Khintchine公式,对于一个任意分布的服务时间T,
    且规定对应分布的服务强度$\rho=\lambda E(t_{all})$,有
    \begin{equation}
        L_s=\rho +\frac{\rho^2 +\lambda^2 Var(T)}{2(1-\rho)}
    \end{equation}
    根据Little法则,可以对平均服务长度与平均等待时长进行换算
    \begin{equation}
        L_s=\lambda W_s
    \end{equation}

    基于以上理论分析,给出我们模型的理论平均队长与理论平均等待时间,
    在考虑随机服务时间为定值的条件下
    \begin{equation}
        \begin{aligned}
            L_s & =\rho +\frac{\rho^2 }{2(1-\rho)} \\
            W_s &=\frac{\rho}{\lambda} +\frac{\rho^2}{2\lambda (1-\rho)}
        \end{aligned}
    \end{equation}
    在考虑随机服务时间为负指数分布的条件下
    \begin{equation}
        \begin{aligned}
            L_s & =\rho +\frac{\rho^2 +\lambda^2 /\mu^2}{2(1-\rho)} \\
            W_s &=\frac{\rho}{\lambda} +\frac{\rho^2 +\lambda^2 /\mu^2}{2\lambda (1-\rho)}
        \end{aligned}
    \end{equation}

\subsubsection{求解方法}
    为了细致的分析这一过程,我们需要求解得到系统的运行特征$P_n(t)$,
    它表示系统在任意时刻t系统中有n个人的概率。
    
    为方便后续的描述,将平均服务时间简写为$\frac{1}{\mu}$,实际为$2t_c+\frac{1}{\mu}$。
    
    先在t时刻与$t+\Delta t$时刻之间的时间段内进行研究。由于在任意时间段内
    一个顾客到达的概率与t无关,所以有一个顾客到来的概率为$\lambda \Delta t$,
    与之对应,没有顾客到来的概率为$1- \lambda \Delta t$。
    
    由于此时将服务时间简化为定值,所以这段时间内一个顾客离去的概率为$\mu \Delta t$,
    没有顾客离去的概率为$1-\mu \Delta t$。
    
    发生两个以上顾客到来或者离去的概率为以上两个概率的平方,所以是$o(\Delta)t$,可以忽略不计。
    
    接下来,在t时刻与$t+\Delta t$时刻之间的时间段内,可能发生四种情况,
    \begin{itemize}
    \item 一个人到来,一个人离去,概率为$\lambda \Delta t \cdot \mu \Delta t$,结果为人数加1。
    \item 一个人到来,没有人离去,概率为$\lambda \Delta t \cdot (1-\mu \Delta) t$,结果为人数不变。
    \item 没有人到来,一个人离去,概率为$(1-\lambda \Delta) t \cdot \mu \Delta t$,结果为人数减1。
    \item 没有人到来,没有人离去,概率为$(1-\lambda \Delta t) \cdot (1-\mu \Delta) t$,结果为人数不变。
    \end{itemize}
    
    
    再考虑在时刻t时,有n-1,n,n+1个顾客时的情形,要想在时刻$t+\Delta t$时有n个人,
    只有三种情况(不考虑两个人以上的变动),
    \begin{itemize}
    \item 在时刻t有n个人,在接下来$\Delta t$时间段内人数不变。
    \item 在时刻t有n-1个人,在接下来$\Delta t$时间段内人数加1。
    \item 在时刻t有n+1个人,在接下来$\Delta t$时间段内人数减1。 
    \end{itemize}
    
    
    综合得到以下状态转移概率公式:
    \begin{equation}
        P_n(t+\Delta t)=P_n(t)(1-\lambda \Delta t-\mu \Delta t)+P_{n+1}(t)(\mu \Delta t) +P_{n-1}(t)(\lambda \Delta t)+o(\Delta t)
    \end{equation}
    
    在这个状态转移概率公式中对t取极限,得到关于$P_n(t)$差分微分方程:
    \begin{equation}
        \frac{d P_n(t)}{dt}=\lambda P_{n-1}(t)+\mu P_{n+1}(t)-(\lambda+\mu )P_n(t)
    \end{equation}
    
    我们只关心系统状态的稳态解,而不关心系统状态的瞬态解,所以只考虑$P_n$,而不是$P_n(t)$。
    
    从而得到一个关于稳态状态之间的差分方程:
    \begin{equation}
        \begin{aligned}
            \lambda P_{n-1}(t)+\mu P_{n+1}(t)-(\lambda+\mu )P_n(t)&=0,n\geq 1 \\
            -\lambda P_{0}+\mu P_{1}&=0
        \end{aligned}
    \end{equation}
    
    以上的差分方程是很好求解的,由于顾客源于排队的空间都设置为无穷大,所以可以解得:
    \begin{equation}
        \begin{aligned}
            P_0 &=1-\rho \\
            P_n &=(1-\rho)\rho^n,n\geq1
        \end{aligned}
    \end{equation}
    其中的$\rho=\frac{\lambda}{\mu}$,它的物理意义为服务强度。
    
    从而可以得到$L_s$的数学期望$L_s=\frac{\rho}{1-\rho}$,这与
    基本理论中的Pollaczek-Khintchine公式结果是一致的。
\subsection{单通道确定性模型}
\subsubsection{理论分析}
首先考虑非高峰期的情况下,使用一个参数为$\lambda_{l}$的泊松过程来表示顾客源,
使用一个常数(平均服务时间)来代替负指数分布的随机服务时长,即$M/D/1/\infty/\infty/FCFS$模型。

使用上述方法得到一个关于稳态状态之间的差分方程:
\begin{equation}
    \begin{aligned}
        \lambda P_{n-1}(t)+\mu P_{n+1}(t)-(\lambda+\mu )P_n(t)&=0,n\geq 1 \\
        -\lambda P_{0}+\mu P_{1}&=0
    \end{aligned}
\end{equation}

这个差分方程表示的各个不同状态之间的转移关系,可以用状态转移图来表示:
\begin{figure}[ht]    
    \centering
    \includegraphics[width=.7\textwidth]{images/transform1.PNG}
    \caption{单通道状态转移图}
    \label{fig:transform1}
\end{figure}
\par 对于默认关门的情况,服务时间的均值$E(t_1)=2t_c+\frac{1}{\mu}$,
方差$Var(t_1)=0$
所以服务强度$\rho_1=2t_c\lambda_l+\frac{\lambda_l}{\mu}$,
由于将服务时间设置为确定值,所以直接使用Pollaczek-Khintchine公式进行求解。
得到理论上的平均队长$L_s=\rho_1+\frac{\rho_1^2}{2(1-\rho_1)}$。
\par 同理,对于默认开门的情况,服务时间是$t_2=t_r+t_{penal}$,
$t_{penal}$表示刷卡失败的惩罚时间,
随机变量$t_{penal}$可以用以下概率公式描述
\begin{equation}
    p(t_{penal})=
    \begin{cases}
        1-p_o,t_{penal}=T_{penal} \\
        p_o,t_{penal}=0
    \end{cases}
\end{equation}
其中$p_o$为开门成功率。$T_{penal}$为惩罚的具体时间大小。
这里的$t_r$与$t_{penal}$是相互独立的,因为后者是否存在仅仅取决于是否刷卡失败,
而这对随机服务时间,即刷卡时间与经过时间,没有任何影响。从而得到总服务时间的期望与方差如下:
\begin{equation}
    \begin{aligned}
        E(t_2) &=\frac{1}{\mu}+(1-p_o)T_{penal} \\
        Var(t_2)&=\frac{1}{\mu^2}+p_o(1-p_o)T_{penal}^2
    \end{aligned}
\end{equation}

所以服务强度$\rho_2=\lambda_l (1-p_o)T_{penal}+\frac{\lambda_l}{\mu}$,
直接运用Pollaczek-Khintchine公式得到理论上的平均队长
$L_s=\rho_2+\frac{\rho_2^2+\lambda^2/\mu^2+\lambda^2 p_o(1-p_o)T_{penal}^2}{2(1-\rho_2)}$
\par 最后将两者比较,得到两种服务方式的平均队长的比值
\begin{equation}
    \begin{aligned}
        \frac{L_{s1}}{L_{s2}}&=\frac{\rho_1+\frac{\rho_1^2}{2(1-\rho_1)}}{\rho_2+\frac{\rho_2^2+\lambda_l^2/\mu^2+\lambda_l^2 p_o(1-p_o)T_{penal}^2}{2(1-\rho_2)}} \\
        &=\frac{(2\rho_1-\rho_1^2)(1-\rho_2)}{(1-\rho_1)[2\rho_2-\rho_2^2+\lambda_l^2/\mu^2+\lambda_l^2 p_o(1-p_o)T_{penal}^2]}
    \end{aligned}
\end{equation}
其中$\rho_1=2t_c\lambda_l+\frac{\lambda_l}{\mu}$,
$\rho_2=\lambda_l (1-p_o)T_{penal}+\frac{\lambda_l}{\mu}$

接下来对结果进行简单的讨论:
\begin{itemize}
    \item 对于服务强度,$\rho_1-\rho_2=\lambda_l(2t_c-(1-p_o)T_{penal})$,
    由于$p_o$是一个很大的值,一般在0.9以上,所以默认关门的服务强度
    比默认关门要大得多,即在人流相同的条件下,默认开门的通行压力要更小。
    \item 对于这个公式,给出我们接下来数值时使用的参数,代入公式进行计算,
    具体取值如下:$t_c=0.5$,$\mu=1$,$p_o=0.9$,$T_{penal}=1$,$\lambda_l=0.2$,
    则$\rho_1=0.4,\rho_2=0.22$,从而$L_{s1}=\frac{3.2}{6},L_{s2}=0.278,\frac{L_{s1}}{L_{s2}}=1.9>1$。
    所以一般默认开门的平均队长是默认关门平均队长的0.52倍。
    也可以根据Little法则算出对应的等待时间比值,与平均队长的比值是一致的。
\end{itemize}

\subsubsection{模型建立}
\par 将校门前的道路网格化,假设行人占据一个网格、自行车占据两个网格。
根据假设,自行车在通过校门的时候会下车推行,
从而行人和自行车在门前的前进速度可以认为相同,设为$1m/s$。
对排队人群进行离散模拟,在每个模拟时间段内,如果行人或自行车前方空格
没有被占用,则前进一格。根据这样的模拟规则,将一个网格的长度设为$1m$。
\par 考虑单通行道的情况,模型可视化展示在图\ref{fig:one-lane-model}中。
\begin{figure}[ht]
    \centering
    \includegraphics[width=0.6\textwidth]{images/cellular_automata_1_lane.png}
    \caption{单通行道模型可视化,其中灰色块表示网格被占用,红色块表示
    该通行者会刷卡失败(或没有校园卡),图中最右部分为校门}
    \label{fig:one-lane-model}
\end{figure}
\newline 其中:
\begin{itemize}
    \item $L_p, L_b$分别表示行人和自行车占用的网格长度
    \item $v_f$表示行人和自行车的前进速度
    \item $t_o, t_c$分别表示校门的开关时间,考虑实际情况,认为$t_c=t_o$
    \item $t_g,t_p$(图中未标注)分别为刷卡时间和通过时间,在确定性模型中
    两者视为定值
\end{itemize}
人和自行车均在道路最左边生成,道路有最大长度,当道路最左侧已经被占用时,
不再生成人。认为人的生成过程是一个泊松过程,对于高人流量和低人流量的时间段
泊松过程的参数取值不同。
\par 刷卡进校过程的拆分及两种开门方式的不同展示在图\ref{fig:process-entering-gate}中。
\begin{figure}[ht]    
    \centering
    \includegraphics[width=.7\textwidth]{images/enter_gate_process.png}
    \caption{刷卡进校过程及两种校门开关方式的异同}
    \label{fig:process-entering-gate}
\end{figure}
在刷卡进校时,对于两种方法做相应的时间消耗分析:
\begin{itemize}
    \item 门保持开放:
    \begin{itemize}
        \item 如果刷卡成功,直接通过校门,需要的时间即为通过校门的时间加上
        刷卡识别的时间$t_p+t_g$。如果上一个人刷卡失败,该时间变为$t_g+t_o+t_p$。
        \item 如果刷卡失败,需要等待门关闭。根据实际生活经验,刷卡失败
        的个体在刷卡过程中一般花费时间也较长(例如询问在哪里刷身份证),同时
        刷卡失败之后,一般要离开队伍,然后登记入校。
        将这两个过程中时间的消耗合计为
        $t_{penal}$,从而刷卡失败消耗的时间为刷卡时间、等待门关闭的时间和
        离开队伍的时间加和$t_g+t_c+t_{penal}$。
    \end{itemize}
    \item 门正常开关:
    \begin{itemize}
        \item 如果刷卡成功,等待门开放后通过校门,下一个在队伍中的人等待门
        关闭后继续刷卡通过。从而一个人需要的时间为$t_g+t_o+t_p+t_c$。
        \item 如果刷卡失败,需要离开队伍。门正常开关时,不用等待门关闭,
        从而所需时间为$t_g+t_{penal}$。
    \end{itemize}
\end{itemize}

\subsubsection{数值模拟}
模拟的时间步长$t^*$取为$0.5s$,根据实际生活经验,约定开门的时间$t_o=t^*$、
行人的刷卡时间$t_g=t^*$、通过时间$t_p=t^*$、门关上的时间$t_c=t_o=t^*$。
如果刷卡失败,耽误的时间$t_{penal}=2t^*$。根据模型分析:
\begin{itemize}
    \item 门保持开放:
    \begin{itemize}
        \item 刷卡成功:耗时$t_p+t_g=2t^*$。
        \item 刷卡失败:耗时$t_g+t_c+t_{penal}=4t^*$。
    \end{itemize}
    \item 门正常开关:
    \begin{itemize}
        \item 刷卡成功:耗时$t_g+t_o+t_p+t_c=4t^*$。
        \item 刷卡失败:耗时$t_g+t_{penal}=3t^*$
    \end{itemize}
\end{itemize}
设置好以上参数以后,进行10000次模拟,每隔4个时间步长$t^*$取样一次,
设置队列容量为40,如果在模拟过程中,队列长度很少达到40,则表示符合队列容量无穷大的假设。
\par 数值模拟结果展示在图\ref{fig:queue-length-time}中,定性观察到
红色曲线几乎都在蓝色曲线上方,从而默认开门模式的队伍平均长度较短,即通行效率较高。
\par 为了定量描述两个模式效率的差距,绘制两模式队长比值-时间曲线在图\ref{fig:length-ratio-time-curve}中,
即$\frac{L_{s1}}{L_{s2}}-t$的关系。这个比值是通过对10000次取样的
队长取平均得到的平均队长,从而能够使得队长的比值趋于稳定。通过观察图像
与定量的数值分析,根据统计学知识,按照置信度为$\alpha=0.90$,计算得到
对应置信区间[0.4516,0.4526]。而上文中理论计算得到的比值为0.52,
可以得出两种模式下,理论分析的定性结论与实验一致,但定量结果略有偏差,这可能由于
理论推导中使用的是队伍极限情况的数据,但是数值模拟的时间长度是有限的。
\begin{figure}
    \centering
    \begin{minipage}[c]{0.45\textwidth}
        \centering
        \includegraphics[width=1.\textwidth]{images/队长-时间曲线.png}
        \subcaption{单通道确定性模型队长-时间曲线,其中红色线表示正常开门方式,蓝色线表示校门保持
        开放}
        \label{fig:queue-length-time}
    \end{minipage}
    \begin{minipage}[c]{0.45\textwidth}
        \centering
        \includegraphics[width=1.\textwidth]{images/队长比例-时间曲线.png}
        \subcaption{单通道确定性模型两模式队长比值-时间曲线}
        \label{fig:length-ratio-time-curve}
    \end{minipage}
    \caption{单通道确定性模型数值模拟}
    \label{fig:one-lane-numerical-result}
\end{figure}
\par 在图\ref{fig:queue-length-time}中观察到,在正常开门的情况下,
队伍人数在到达一个极值后存在直线下降的情况,这与生活经验有差别。
出现这种情况可能由于,模型中规定如果道路已经被填满则不会再生成人,这与实际
情况是不一致的。所以在之后的模型中,用一个``缓冲区''
来存储当道路填满时生成的人,更好地拟合实际情况。