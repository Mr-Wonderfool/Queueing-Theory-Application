\section{基于模型的建议}
\par 在非高峰期,根据模型结果,默认关门的平均队长比大约是默认开门的两倍,
通行效率显著低于后者,几乎没有可比性。在高分期且刷卡失败率较高时,
两种通行模式的的平均队长之比趋近于1,表明两者的效率相近。而这时,
默认关门比默认开门的安全性大大提高。所以在人流较大且刷卡失败率
高的情况下,应当使用默认关门的模式。
\par 就同济大学而言,在特殊时期,例如樱花季、部分法定假期等,
人流量显著提高,且主要通行人员的身份从学生转为游客,刷卡成功率大大降低,
而且安全性也有所降低。这时使用默认关门的模式,在通行效率上没有太大的影响,
而在安全性上得到提高。而在平常时期,出入主体为学生,可以使用默认开门
的模式,提高通行效率。
\par 除此以外,此模式还可以推广到其他出入系统,如地铁站。
选取何种模式应该取决于具体情况,通过对地区的交通数据分析,统计得到
人流量、通行人主要成分、通行时间段等因素,综合判断通行效率与安全性,
给出对应开门方式。