\documentclass{article}
\usepackage{ctex}
\usepackage{amsmath}
\usepackage{booktabs}
\usepackage{mathtools}
\usepackage{amsfonts}
\usepackage{amsthm}
\usepackage{tabularx}

\begin{document}

\section{符号说明}
\begin{table}[ht]
    \centering
    \caption{符号说明表格}\label{tab:symbols}
    \begin{tabular}{c>{\centering\arraybackslash}p{0.7\textwidth}>
        {\centering\arraybackslash}p{0.1\textwidth}}
        \toprule[1.5pt]
        符号 & 符号说明 & 量纲\\
        \midrule[1pt]
        $ t_g $ & 刷卡时间 & $s$\\
        $ t_o $ & 开门时间 & $s$\\
        $ t_p $ & 经过时间 & $s$\\
        $ t_c $ & 关门时间 & $s$\\
        $ t_s $ & 随机服务时间 & $s$\\
        $ t_{all} $ & 总服务时间 & $s$\\
        $ v_f$ & 队伍前进速度 & $m/s$\\
        $ p_s$ & 换道的概率 & $-$ \\
        $C_{source}$ & 顾客源的容量 & $-$ \\
        $C_{queue}$ & 队列的空间容量 & $-$ \\
        $ l_b$ & 自行车占据空间 & $m$\\
        $ l_p$ & 普通行人占据空间 & $m$\\
        $ L_s$ & 服务系统中的总共人数 & $-$\\
        $ L_q$ & 服务系统队列中的人数 & $-$\\
        $ W_s$ & 服务系统中平均等待时长 & $s$\\
        $ W_q$ & 服务系统队列平均等待时长 & $s$\\
        $ \lambda_l$ & 非高峰期时人群的平均到达率 & $-$\\
        $ \lambda_h$ & 高峰期时人群的平均到达率 & $-$\\
        $ \mu $ & 随机服务时的单位时间内的服务人数 & $-$\\
        $ \rho $ & 服务强度 & $-$\\
        \bottomrule[1.5pt]
    \end{tabular}
\end{table}

\section{问题分析}
从理论的角度,通过排队论的方法对问题进行建模。需要考虑到服务对象,服务规则来构建排队模型,使用
常见的系统平均队长$ L_s$,平均等待时间$W_s$,服务强度$\rho$等概念来对比两种开门模式的效率。
另一方面,从模拟的角度,基于理论分析的顾客、服务模型,建立元胞自动机模拟
排队过程,计算最终结果,与理论计算值进行对比。

由于两个开门模式的差别在于服务模式的不同,而对于顾客源,两者是一致的。
因此,可以将顾客源进行适当的简化,而突出服务模式的对排队系统的影响。

另一方面,研究由浅入深,逐步增加模型的复杂度,从而使得模型的效果更接近实际。
\begin{description}
    \item[模型1] 在时间上,只考虑非高峰期,在通道上只考虑一个校门的情形,在服务系统方面将随机服务时间用
    固定数值(均值)代替,从而简化模型。
    \item[模型2] 在时间上,只考虑非高峰期,在通道上只考虑一个校门的情形,在服务系统方面,随机服务时间用
    负指数分布来描述,从而细化对服务系统的刻画。
    \item[模型3] 在时间上,既考虑非高峰期,又考虑高峰期,在通道上考虑多个校门的情形,并且
    顾客流可以在不同队列中按照一定概率转移,在服务系统方面,随机服务时间用
    负指数分布来描述,得到最终的效率对比结果。
\end{description}

以下对这些问题一一建模。

\subsection{服务规则分析}
	在两种方案中,无论默认状态是开门还是关门,顾客的服务时间都是相对固定的。

    对于默认关门的情况,顺利刷卡的服务时间为刷卡时间$t_{g}$,开门时间$t_{o}$,经过时间$t_{p}$
    与关门时间$t_{c}$组成。其中开门、关门时间即为定值,而且是相等的,
    相加作为固定服务时间$2t_{c}$。而刷卡时间、经过时间有一定的波动,
    对于不同年龄、不同身份、不同交通工具的人有较大的差异。
    例如,骑行单车入校的人由于要控制单车,取出与放回校园卡
    的时间比步行要长;年龄较大的人通过时间比年轻人要长;通过时看手机
    的人,通过时间比正常通行的人要长。将这两个服务时间之和即为随机服务时间$t_{r}$
    
    但总的来说,校门的随机服务时间有一定的共性。首先,服务时间必然是正数;其次大多数人的
    服务时间都较为稳定,服务时间极长的人很少。为了描述这种变化,使用常见的负指数分布
    来描述校门的服务过程。

    最后将固定服务时间与随机服务时间相加即为总服务时间$t_{all}=2t_{c}+t_{r}$.

    而对于刷卡失败的情况,在非高峰期时,队列稀疏的情况下,对正常队列的影响不大,
    但是在高峰期时,由于进出压力大,人们在刷卡的流程中形成惯性,而刷卡失败会
    导致队列结果的破坏与前后人流的拥堵,从而大大降低通过效率,
    甚至导致整条队列的长期停滞。对刷卡失败的情况,需要细致的建模。

    对于默认开门的情况,顺利刷卡的服务时间为刷卡时间与经过时间,也可以视为定值。
    但是对于刷卡失败的人员,需要在保安的协助下进行登记之后才能入内,而下一个
    人的经过时间需要加上开门时间。

\subsection{顾客源分析}
在我们的问题中,服务对象为进出校的学生以及其他人员。在这些服务对象中,又分为步行
与单车两重出行方式。在分布时间上,又分为高峰期与非高峰期。总的来说需要对不同时期、不同方式、
不同身份的服务对象分别进行建模,以下是具体分析过程

\subsubsection{学生}
	学生群体是所有服务对象中规模最大,成分最复杂的情况,也是开门方式能够影响的主要人员,需
    要进行详细的建模。

	从时间分布的角度来说,由于学生群体在高峰期与非高峰期上具有明显的差异。在非高峰期时,学生
    进出校门的时间随机性强,且较为稀疏。进一步分析,在不相互重叠的时间段内顾客到达的数量
    是相互独立的;且在任意时间段内,到达的概率与时间无关,因为在非高峰期的出行没有显著规律;
    最后,由于非高峰期人群的通行具有稀疏性,所以在短时间内有多人进入的概率极小。
    
    综合以上原因,顾客源适合用标准泊松流来建模。而且由于几乎不存在过于拥挤的
    可能,因此也不考虑容量限制。即$M/G/1/\infty/\infty/FCFS$模型。

    在高峰期,人员流动迅速,排队长度较长,但是高峰期的流通人数是由限制的,即顾客源的容
    量是有限的,将此时顾客源的容量记为$C_{source}$。而且在校门口排队的空间容量是有限的,
    在高峰期容易被填满,将此时的总容量记为$C_{queue}$。但总体来说,顾
    客到来的间隔时间还是可以服从一个均值较小的泊松分布,这个均值比非高峰期要小得多。综合
    起来,高峰期的学生进校排队模型可以简化为$M/G/3/C_{queue}/C_{source}/FCFS$
    此外,由于步行与单车在排队是占据的空间有多不同,所以最后的空间限制并非直接作用于人数,
    而是需要通过作用于空间,间接控制人数。

    从刷卡成功率的角度分析,学生群体的刷卡成功率很高。

\subsubsection{其他}
    对于其他人群,在进校时间上没有明显的高峰期与非高峰期之分,所以用统一的泊松过程模拟即可。
    而且由于其他人群的进出校需求与非高峰期的学生大致相当,可以直接与学生统一。
    对于交通方式来说,主要是步行,而骑单车的人数较少,故简化为全步行。
    在刷卡成功率方面,这个群体的成功率要比学生群体更低。如果与学生顾客源
    统一,则需要按照双方人数加权平均,得到综合的刷卡成功率。


\section{模型求解}

\subsection{理论求解}

\subsubsection{基本理论}
    首先从理论上,基于排队论的基本公式与概率论相关知识,在理论上对两种开门模型
    进行理论计算,得到平均队长与平均等待时间作为评价通行效率的指标。

    从顾客源的角度来说,顾客源的输入为泊松过程,泊松过程的概率表达式如下

    \begin{equation}
        P_n(t)=\frac{e^{-\lambda t}(\lambda t)^n}{n!},t>0,n \in \mathbb{N}
    \end{equation}
    $\lambda$表示单位时间内平均到达的顾客数,
    对于非高峰期与高峰期的顾客源,分别使用不同的$\lambda_l,\lambda_h$来描述。

    接下来,基于python语言,按照泊松过程的概率公式,随机生成顾客队列,
    并让队列以$v_f=1 m/s$的速度前进。

    对于到达校门的人,开始进行服务。服务时间由固定服务时间与随机服务时间相加得到,
    即$t_{all}=2t_{c}+t_{r}$。$t_r$服从负指数分布,概率密度与分布函数如下:
    \begin{equation}
        \begin{aligned}
            f(t) &=\mu e^{-\mu t},  \\
            F(t) &=1-e^{-\mu t},t>0
        \end{aligned}
    \end{equation}
    求解服务时间$t_r$的数学期望$E(t_r)=\frac{1}{\mu}$即为平均服务时间,
    因此$\mu$的意义就是单位时间内的平均服务人数,也就是校门的平均通过速率。
    在加上固定服务时间$2t_c$,得到总服务时间的数学期望$E(t_{all})=2t_c+\frac{1}{\mu}$,
    方差$ Var(t_{all})=Var(t_r)=\frac{1}{\mu^2}$。
    对于精度要求不高的模型,也可以使用一个常数(平均服务时间)来代替,
    即数学期望$E(t_{all})=2t_c+\frac{1}{\mu}$,
    方差$ Var(t_{all})=0$。

    有了顾客源与服务时间的随机分布模型,可以服务定义强度
    $\rho=\lambda E(t_{all})=2t_c \lambda+\frac{\lambda}{\mu}$

    在这里假设服务强度$\rho<1$,而这个假设也是合理的,因为否则会导致队伍长度发散。
    使用排队论中的经典公式Pollaczek-Khintchine公式,对于一个任意分布的服务时间T,
    且规定对应分布的服务强度$\rho=\lambda E(t_{all})$,有
    \begin{equation}
        L_s=\rho +\frac{\rho^2 +\lambda^2 Var(T)}{2(1-\rho)}
    \end{equation}
    根据Little法则,可以对平均服务长度与平均等待时长进行换算
    \begin{equation}
        L_s=\lambda W_s
    \end{equation}

    基于以上理论分析,给出我们模型的理论平均队长与理论平均等待时间,
    在考虑随机服务时间为定值的条件下
    \begin{equation}
        \begin{aligned}
            L_s & =\rho +\frac{\rho^2 }{2(1-\rho)} \\
            W_s &=\frac{\rho}{\lambda} +\frac{\rho^2}{2\lambda (1-\rho)}
        \end{aligned}
    \end{equation}
    在考虑随机服务时间为负指数分布的条件下
    \begin{equation}
        \begin{aligned}
            L_s & =\rho +\frac{\rho^2 +\lambda^2 /\mu^2}{2(1-\rho)} \\
            W_s &=\frac{\rho}{\lambda} +\frac{\rho^2 +\lambda^2 /\mu^2}{2\lambda (1-\rho)}
        \end{aligned}
    \end{equation}

\subsubsection{模型1}
首先考虑非高峰期的情况下,使用一个参数为$\lambda_{l}$的泊松过程来表示顾客源,
使用一个常数(平均服务时间)来代替负指数分布的随机服务时长,即$M/D/1/\infty/\infty/FCFS$模型。

为了细致的分析这一过程,我们需要求解得到系统的运行特征$P_n(t)$,
它表示系统在任意时刻t系统中有n个人的概率。

为方便后续的描述,将平均服务时间简写为$\frac{1}{\mu}$,实际为$2t_c+\frac{1}{\mu}$。

先在t时刻与$t+\Delta t$时刻之间的时间段内进行研究。由于在任意时间段内
一个顾客到达的概率与t无关,所以有一个顾客到来的概率为$\lambda \Delta t$,
与之对应,没有顾客到来的概率为$1- \lambda \Delta t$。

由于此时将服务时间简化为定值,所以这段时间内一个顾客离去的概率为$\mu \Delta t$,
没有顾客离去的概率为$1-\mu \Delta t$。

发生两个以上顾客到来或者离去的概率为以上两个概率的平方,所以是$o(\Delta)t$,可以忽略不计。

接下来,在t时刻与$t+\Delta t$时刻之间的时间段内,可能发生四种情况,
\begin{itemize}
\item 一个人到来,一个人离去,概率为$\lambda \Delta t \cdot \mu \Delta t$,结果为人数加1。
\item 一个人到来,没有人离去,概率为$\lambda \Delta t \cdot (1-\mu \Delta) t$,结果为人数不变。
\item 没有人到来,一个人离去,概率为$(1-\lambda \Delta) t \cdot \mu \Delta t$,结果为人数减1。
\item 没有人到来,没有人离去,概率为$(1-\lambda \Delta t) \cdot (1-\mu \Delta) t$,结果为人数不变。
\end{itemize}


再考虑在时刻t时,有n-1,n,n+1个顾客时的情形,要想在时刻$t+\Delta t$时有n个人,
只有三种情况(不考虑两个人以上的变动),
\begin{itemize}
\item 在时刻t有n个人,在接下来$\Delta t$时间段内人数不变。
\item 在时刻t有n-1个人,在接下来$\Delta t$时间段内人数加1。
\item 在时刻t有n+1个人,在接下来$\Delta t$时间段内人数减1。 
\end{itemize}


综合得到一下状态转移概率公式:
\begin{equation}
    P_n(t+\Delta t)=P_n(t)(1-\lambda \Delta t-\mu \Delta t)+P_{n+1}(t)(\mu \Delta t) +P_{n-1}(t)(\lambda \Delta t)+o(\Delta t)
\end{equation}

在这个状态转移概率公式中对t取极限,得到关于$P_n(t)$差分微分方程:
\begin{equation}
    \frac{d P_n(t)}{dt}=\lambda P_{n-1}(t)+\mu P_{n+1}(t)-(\lambda+\mu )P_n(t)
\end{equation}

我们只关心系统状态的稳态解,而不关心系统状态的瞬态解,所以只考虑$P_n$,而不是$P_n(t)$。

从而得到一个关于稳态状态之间的差分方程:
\begin{equation}
    \begin{aligned}
        & \lambda P_{n-1}(t)+\mu P_{n+1}(t)-(\lambda+\mu )P_n(t)=0,n\geq 1 \\
        & -\lambda P_{0}+\mu P_{1}=0
    \end{aligned}
\end{equation}

这个差分方程表示的各个不同状态之间的转移关系,可以用状态转移图来表示:


\subsubsection{模型2}
考虑非高峰期的情况下,使用一个参数为$\lambda_{l}$的泊松过程来表示顾客源,
使用负指数分布的随机服务时长,即$M/G/1/\infty/\infty/FCFS$模型。
其中使用的是G而不是M是因为总的服务时长$t_{all}=2t_{c}+t_{r}$,是一个
符合负指数分布的变量加上一个常量,并不是严格的负指数分布,所以用一般分布G表示。

对本模型的求解,主要按照按模型1的求解方法,先建立对应稳态之间的差分方程:


以下是不同时间点的状态转移关系:

\subsubsection{模型3}
考虑非高峰期高峰期的情况下,使用一个参数为$\lambda_{l},\lambda_{h}$的泊松过程来表示顾客源,
使用负指数分布的随机服务时长,考虑多通道且允许相互转移的过程,
即$M/G/3/C_{queue}/C_{source}/FCFS$模型。

以下是不同时间点的状态转移关系:

\subsection{模拟求解}

\subsubsection{模型1}

\subsubsection{模型2}

\subsubsection{模型3}

\end{document}