\section{模型求解}

\subsection{理论求解}

\subsubsection{基本理论}
    首先从理论上,基于排队论的基本公式与概率论相关知识,在理论上对两种开门模型
    进行理论计算,得到平均队长与平均等待时间作为评价通行效率的指标。

    从顾客源的角度来说,顾客源的输入为泊松过程,泊松过程的概率表达式如下

    \begin{equation}
        P_n(t)=\frac{e^{-\lambda t}(\lambda t)^n}{n!},t>0,n \in \mathbb{N}
    \end{equation}
    $\lambda$表示单位时间内平均到达的顾客数,
    对于非高峰期与高峰期的顾客源,分别使用不同的$\lambda_l,\lambda_h$来描述。

    接下来,基于python语言,按照泊松过程的概率公式,随机生成顾客队列,
    并让队列以$v_f=1 m/s$的速度前进。

    对于到达校门的人,开始进行服务。服务时间由固定服务时间与随机服务时间相加得到,
    即$t_{all}=2t_{c}+t_{r}$。$t_r$服从负指数分布,概率密度与分布函数如下:
    \begin{equation}
        \begin{aligned}
            f(t) &=\mu e^{-\mu t},  \\
            F(t) &=1-e^{-\mu t},t>0
        \end{aligned}
    \end{equation}
    求解服务时间$t_r$的数学期望$E(t_r)=\frac{1}{\mu}$即为平均服务时间,
    因此$\mu$的意义就是单位时间内的平均服务人数,也就是校门的平均通过速率。
    在加上固定服务时间$2t_c$,得到总服务时间的数学期望$E(t_{all})=2t_c+\frac{1}{\mu}$,
    方差$ Var(t_{all})=Var(t_r)=\frac{1}{\mu^2}$。
    对于精度要求不高的模型,也可以使用一个常数(平均服务时间)来代替,
    即数学期望$E(t_{all})=2t_c+\frac{1}{\mu}$,
    方差$ Var(t_{all})=0$。

    有了顾客源与服务时间的随机分布模型,可以服务定义强度
    $\rho=\lambda E(t_{all})=2t_c \lambda+\frac{\lambda}{\mu}$

    在这里假设服务强度$\rho<1$,而这个假设也是合理的,因为否则会导致队伍长度发散。
    使用排队论中的经典公式Pollaczek-Khintchine公式,对于一个任意分布的服务时间T,
    且规定对应分布的服务强度$\rho=\lambda E(t_{all})$,有
    \begin{equation}
        L_s=\rho +\frac{\rho^2 +\lambda^2 Var(T)}{2(1-\rho)}
    \end{equation}
    根据Little法则,可以对平均服务长度与平均等待时长进行换算
    \begin{equation}
        L_s=\lambda W_s
    \end{equation}

    基于以上理论分析,给出我们模型的理论平均队长与理论平均等待时间,
    在考虑随机服务时间为定值的条件下
    \begin{equation}
        \begin{aligned}
            L_s & =\rho +\frac{\rho^2 }{2(1-\rho)} \\
            W_s &=\frac{\rho}{\lambda} +\frac{\rho^2}{2\lambda (1-\rho)}
        \end{aligned}
    \end{equation}
    在考虑随机服务时间为负指数分布的条件下
    \begin{equation}
        \begin{aligned}
            L_s & =\rho +\frac{\rho^2 +\lambda^2 /\mu^2}{2(1-\rho)} \\
            W_s &=\frac{\rho}{\lambda} +\frac{\rho^2 +\lambda^2 /\mu^2}{2\lambda (1-\rho)}
        \end{aligned}
    \end{equation}

\subsubsection{模型1}
首先考虑非高峰期的情况下,使用一个参数为$\lambda_{l}$的泊松过程来表示顾客源,
使用一个常数(平均服务时间)来代替负指数分布的随机服务时长,即$M/D/1/\infty/\infty/FCFS$模型。

为了细致的分析这一过程,我们需要求解得到系统的运行特征$P_n(t)$,
它表示系统在任意时刻t系统中有n个人的概率。

为方便后续的描述,将平均服务时间简写为$\frac{1}{\mu}$,实际为$2t_c+\frac{1}{\mu}$。

先在t时刻与$t+\Delta t$时刻之间的时间段内进行研究。由于在任意时间段内
一个顾客到达的概率与t无关,所以有一个顾客到来的概率为$\lambda \Delta t$,
与之对应,没有顾客到来的概率为$1- \lambda \Delta t$。

由于此时将服务时间简化为定值,所以这段时间内一个顾客离去的概率为$\mu \Delta t$,
没有顾客离去的概率为$1-\mu \Delta t$。

发生两个以上顾客到来或者离去的概率为以上两个概率的平方,所以是$o(\Delta)t$,可以忽略不计。

接下来,在t时刻与$t+\Delta t$时刻之间的时间段内,可能发生四种情况,
\begin{itemize}
\item 一个人到来,一个人离去,概率为$\lambda \Delta t \cdot \mu \Delta t$,结果为人数加1。
\item 一个人到来,没有人离去,概率为$\lambda \Delta t \cdot (1-\mu \Delta) t$,结果为人数不变。
\item 没有人到来,一个人离去,概率为$(1-\lambda \Delta) t \cdot \mu \Delta t$,结果为人数减1。
\item 没有人到来,没有人离去,概率为$(1-\lambda \Delta t) \cdot (1-\mu \Delta) t$,结果为人数不变。
\end{itemize}


再考虑在时刻t时,有n-1,n,n+1个顾客时的情形,要想在时刻$t+\Delta t$时有n个人,
只有三种情况(不考虑两个人以上的变动),
\begin{itemize}
\item 在时刻t有n个人,在接下来$\Delta t$时间段内人数不变。
\item 在时刻t有n-1个人,在接下来$\Delta t$时间段内人数加1。
\item 在时刻t有n+1个人,在接下来$\Delta t$时间段内人数减1。 
\end{itemize}


综合得到一下状态转移概率公式:
\begin{equation}
    P_n(t+\Delta t)=P_n(t)(1-\lambda \Delta t-\mu \Delta t)+P_{n+1}(t)(\mu \Delta t) +P_{n-1}(t)(\lambda \Delta t)+o(\Delta t)
\end{equation}

在这个状态转移概率公式中对t取极限,得到关于$P_n(t)$差分微分方程:
\begin{equation}
    \frac{d P_n(t)}{dt}=\lambda P_{n-1}(t)+\mu P_{n+1}(t)-(\lambda+\mu )P_n(t)
\end{equation}

我们只关心系统状态的稳态解,而不关心系统状态的瞬态解,所以只考虑$P_n$,而不是$P_n(t)$。

从而得到一个关于稳态状态之间的差分方程:
\begin{equation}
    \begin{aligned}
        & \lambda P_{n-1}(t)+\mu P_{n+1}(t)-(\lambda+\mu )P_n(t)=0,n\geq 1 \\
        & -\lambda P_{0}+\mu P_{1}=0
    \end{aligned}
\end{equation}

这个差分方程表示的各个不同状态之间的转移关系,可以用状态转移图来表示:


\subsubsection{模型2}
考虑非高峰期的情况下,使用一个参数为$\lambda_{l}$的泊松过程来表示顾客源,
使用负指数分布的随机服务时长,即$M/G/1/\infty/\infty/FCFS$模型。
其中使用的是G而不是M是因为总的服务时长$t_{all}=2t_{c}+t_{r}$,是一个
符合负指数分布的变量加上一个常量,并不是严格的负指数分布,所以用一般分布G表示。

对本模型的求解,主要按照按模型1的求解方法,先建立对应稳态之间的差分方程:


以下是不同时间点的状态转移关系:

\subsubsection{模型3}
考虑非高峰期高峰期的情况下,使用一个参数为$\lambda_{l},\lambda_{h}$的泊松过程来表示顾客源,
使用负指数分布的随机服务时长,考虑多通道且允许相互转移的过程,
即$M/G/3/C_{queue}/C_{source}/FCFS$模型。

以下是不同时间点的状态转移关系: